\documentclass{article}
% \usepackage[UTF8]{ctex}
\usepackage{ctex}
\usepackage{amsmath,amssymb}
\usepackage{authblk}
\usepackage{xcolor} 
\usepackage{listings}
\lstset{
    basicstyle=\tt\small,
    columns=flexible,
    breaklines=true,
	tabsize=4,
    xleftmargin=2em, %整体距左侧边线的距离为2em
	numbers=left,
    numberstyle=\tt\tiny,
    frame=single,
	backgroundcolor=\color[RGB]{225,225,225}  %代码背景色
}

\begin{document}

\begin{lstlisting}
nohup command
\end{lstlisting}

执行命令,命令不挂起,退出终端后仍然继续执行;
\begin{lstlisting}
command &
\end{lstlisting}
后台执行,允许输入等操作,退出终端后就终止,所以一般用:
\begin{lstlisting}
nohup command &
\end{lstlisting}
\begin{lstlisting}
nohup command
\end{lstlisting}
默认输出到nohup.out文件。
> 是输出重定向符号, < 是输入重定向符号。
\begin{lstlisting}
nohup command >filename
\end{lstlisting}
实现输出到特定文件。
0表示标准输入,1表示标准输出,2表示标准错误输出。
2>\&1 意为将标准错误重定向为标准输出,相当于合并正常输出和报错。
\&表示将1识别为标准输出的含义,而不是文件1。
/dev/null 是预留的空文件,向其中的所有输出都会消失。
\begin{lstlisting}
command >/dev/null 2>&1
\end{lstlisting}
的作用就是将标准输出屏蔽,标准错误重定向到标准输出(即也屏蔽标准错误)。
\begin{lstlisting}
command </dev/null
\end{lstlisting}
的作用是如果命令需要一个输入,则输入空值,这样可以避免出现多余的输入框。

这样就清楚以下语句的含义了:
\begin{lstlisting}
nohup matlab -nodisplay -r xxx >outfile.txt </dev/null 2>&1 &
\end{lstlisting}


conda 配置不同版本的python环境

查询环境列表
\begin{lstlisting}
conda env list
\end{lstlisting}
其中base为默认环境,一般是最新的python,不要改动。

配置新环境,例:
\begin{lstlisting}
conda create -n py37 python=3.7
\end{lstlisting}
配置好后
\begin{lstlisting}
conda init
\end{lstlisting}
之后重启终端。
激活环境用
\begin{lstlisting}
conda activate py37
\end{lstlisting}
切换回默认环境(base)用
\begin{lstlisting}
conda deactivate
\end{lstlisting}
新环境是空的,需要安装各类包。

在对应环境中启动jupyter notebook,但是使用的包还是base里的,解决方法是在启用的虚拟环境中安装
\begin{lstlisting}
conda install nb_conda
\end{lstlisting}
再在对应环境中启动jupyter notebook。
对于新版本python,nb\_conda不再支持,要自行创建jupyter notebook的内核。
激活相应环境,例:
\begin{lstlisting}
conda activate py312
\end{lstlisting}
后面都是在这一环境中进行。
安装需要的组件
\begin{lstlisting}
conda install ipykernel
\end{lstlisting}
注册内核
\begin{lstlisting}
python -m ipykernel install --user --name py312 --display-name "python 3.12"
% --user 表示只对该用户安装,“py312”替换为对应的环境名
\end{lstlisting}
再启动jupyter notebook就可以选择需要的内核了。

删除不需要的环境:
\begin{lstlisting}
conda remove --name old_env --all
% old_env 是待删除的环境的名称
\end{lstlisting}
这也会删除这个环境下安装的所有库。

已安装CUDA和cuDNN,目录都在,但是加载不了,有可能是环境变量没有设置好。
在对应用户终端输入
\begin{lstlisting}
echo $CUDA_HOME
echo $LD_LIBRARY_PATH
\end{lstlisting}
如果没有返回就说明没设置。
下面开始设置,在需要的用户的终端输入
\begin{lstlisting}
vim ~/.bashrc
\end{lstlisting}
按i键开始编辑,在最后添加
\begin{lstlisting}
export CUDA_HOME=/usr/local/cuda-x.x
export PATH=$CUDA_HOME/bin:$PATH
export LD_LIBRARY_PATH=$CUDA_HOME/lib64:$LD_LIBRARY_PATH
export LD_LIBRARY_PATH=/usr/local/cuda-x.x/lib64:$LD_LIBRARY_PATH
% (这里最后两行是不是重复了?)
% cuda-x.x是cuda安装的文件夹,例:cuda-12.8。
\end{lstlisting}
编辑好后按esc,键盘输入:wq回车,退出编辑器。在终端输入
\begin{lstlisting}
source ~/.bashrc
\end{lstlisting}
应用编辑,再检查
\begin{lstlisting}
echo $CUDA_HOME
echo $LD_LIBRARY_PATH
\end{lstlisting}
有值就行。

\end{document}